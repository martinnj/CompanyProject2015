\documentclass[12pt,aspectratio=1610]{beamer}
% adjust the aspect ratio to match the local hardware

\usepackage{pslatex}
%\usepackage{lmodern}

\usetheme[
	unit=ics,
	sidebar=0.18\paperwidth,
	style=standard,
	footstyle=low,
	%TPlrimage=input/wave.jpg,
	]{Frederiksberg}

% design
\definecolor{kug}{rgb}{0.20,0.36,0.24}
\setbeamercovered{transparent}


% fonts and symbols
\usepackage{textcomp}
\usepackage[utf8]{inputenc}
\usepackage[T1]{fontenc}
\usepackage{gensymb}
\DeclareUnicodeCharacter{B0}{\textdegree}
\usepackage{amsmath}
\usepackage{amsfonts}
\usepackage{amssymb}
\usepackage[english]{babel}
\usepackage{csquotes}
\usepackage{siunitx}
\usepackage[export]{adjustbox}

% other stuff
\usepackage{enumerate}
\usepackage{datetime} % useful for \today macro

% Displays a centered image with caption and label.
% usage: \graphicc{width}{file}{caption}{label}
\newcommand{\graphicc}[4]{\begin{figure}[H] \centering
            \includegraphics[width={#1\textwidth}, keepaspectratio=true]{{#2}}
            \caption{{#3}} \label{#4} \end{figure}}

%\input{input/biber_bibliography}
%\bibliography{my_bibliography}

% graphicx settings
%\graphicspath{
%	{some/path/where/my/figures/are/}
%	{another/path/}
%	}
%\DeclareGraphicsExtensions{.pdf,.png,.jpg}

%%%%%%%%%%%%%%%%%%%%%%%%%%%%%%%%%%%%%%%%%%%%%%%%%%%%%%%%%%%%%%%%%%%%%%%%%%%%%%%%

\title[Company Project]{User Behavior Analysis Using Decision Trees}

\author{Martin Nicklas Jørgensen}
	
\institute{Department of Computer Science}

\date{\today}

%%%%%%%%%%%%%%%%%%%%%%%%%%%%%%%%%%%%%%%%%%%%%%%%%%%%%%%%%%%%%%%%%%%%%%%%%%%%%%%%

\begin{document}


\frame[plain]{\titlepage}

\section{Disposition}
\begin{frame}{Disposition}
	\begin{itemize}
		\item Introduction 
		\item Problem Analysis
		\item Results \& Conclusion
		\item Questions
	\end{itemize}
\end{frame}

%%%%%%%%%%%%%%%%%%%%%%%%%%%%%%%%%%%%%%%%%%%%%%%%%%%%%%%%%%%%%%%%%%%%%%%%%%%%%%%%
\section{Introduction}

\begin{frame}{Simplesite ApS}{Introduction}
	\begin{itemize}
		\item Hosts and sells a website CMS.
		\item Founded in 2001 as \textit{Elk Consulting ApS}, later \textit{123hjemmeside ApS} and now \textit{Simplesite ApS}.
		\item 20-49 employees globally.
		\item $400.000$ new websites every month.
		\item $80.000$ paying subscribers.
	\end{itemize}
\end{frame}

\begin{frame}{Problem Description}{Introduction}
	\begin{itemize}
		\item After the change to freemium, a huge increase in new free users was observed.
		\item The number of paying users did not increase comparetively.
		\item Many users stop being active after a few days.
		\item Using Data Science, is it possible to find a pattern in how users who stay are using the site?
	\end{itemize}
\end{frame}


%%%%%%%%%%%%%%%%%%%%%%%%%%%%%%%%%%%%%%%%%%%%%%%%%%%%%%%%%%%%%%%%%%%%%%%%%%%%%%%%
\section{Problem Analysis}

\begin{frame}{CRSIP-DM}{Problem Analysis}
	\graphicc{0.55}{img/wiki_crisp}{Diagram of the CRISP-DM method. Image source: Wikimedia Foundation.}{fig:crisp}
\end{frame}

\begin{frame}{Business Understanding}{Problem Analysis}
	\begin{itemize}
		\item Some users do not stay active for very long, even on the free product.
		\item Can we figure out what makes users stay?
	\end{itemize}
\end{frame}

\begin{frame}{Data Understanding \& Preparation}{Problem Analysis}
	\begin{itemize}
		\item \texttt{EngagementData} datasets.
		\item \texttt{CustomerJourney} datasets.
		\item Features are removed from the datasets if they are derivative or not relevant.
		\item Datasets are merged into a single dataset.
		\item Data from Sep. 2015 are used for training. ($463716$ observations)
		\item Data from Oct. 2015 are used for test. ($495390$ observations)
		\item Final dataset have 15 features one of which is the target variable \textit{iscjretained}.
	\end{itemize}
\end{frame}

\begin{frame}{Modelling - Tree Type}{Problem Analysis}
	\begin{table}[H]
	    \centering
	    \begin{tabular}{l|l|l}
	        \textbf{Max Depth} & \textbf{\texttt{rpart} Accuracy} & \textbf{\texttt{ctree} Accuracy} \\ \hline
	        \textit{4}         & $94.2799$ \%                     & $94.27990$ \%                    \\
	        \textit{8}         & $94.2799$ \%                     & $94.31958$ \%                    \\
	        \textit{12}        & $94.2799$ \%                     & $94.36638$ \%
	    \end{tabular}
	    \caption{The mean accuracy for the different 5-fold cross validation runs.}
	    \label{tab:treecompare}
	\end{table}
\end{frame}

\begin{frame}{Modelling - Formula \& Depth}{Problem Analysis}
	\begin{table}[H]
	    \centering
	    \begin{tabular}{l|l|l}
	        \textbf{Formula}                              & \textbf{Max Depth} & \textbf{Mean Accuracy} \\ \hline
	        \texttt{iscjretained \~{} .}                  & $4$                & $94.27990$ \%          \\
	                                                      & $6$                & $94.29672$ \%          \\ % <-- second highest accuracy
	                                                      & $8$                & $94.31958$ \%          \\ \hline % <-- highest accuracy
	        \texttt{iscjretained \~{} edits14}            & $4$                & $93.50055$ \%          \\
	                                                      & $6$                & $93.50227$ \%          \\
	                                                      & $8$                & $93.50119$ \%          \\ \hline
	        \texttt{iscjretained \~{} logins14}           & $4$                & $94.27990$ \%          \\
	                                                      & $6$                & $94.27990$ \%          \\
	                                                      & $8$                & $94.27990$ \%          \\ \hline
	        \texttt{iscjretained \~{} edits14 + logins14} & $4$                & $94.27990$ \%          \\
	                                                      & $6$                & $94.28465$ \%          \\
	                                                      & $8$                & $94.29414$ \%
	    \end{tabular}
	    \caption{Mean accuracy of different formulas and tree depths using 5-fold cross
	        validation.}
	    \label{tab:formulacompare}
	\end{table}
\end{frame}

\begin{frame}{Evaluation - Dataset Bias}{Problem Analysis}
	\begin{table}[H]
	  \centering
	  \begin{tabular}{lll}
	    \textbf{Dataset}  & \texttt{TRUE} & \texttt{FALSE} \\ \hline
	    \textit{Training} & $30358$       & $433358$       \\
	    \textit{Test}     & $40731$       & $454659$       \\
	    \textit{Equal}    & $30358$       &  $30358$
	  \end{tabular}
	  \caption{The distribution of the \textit{iscjretained} target variable classes
	    in the different datasets.}
	  \label{tab:datasetretention}
	\end{table}
\end{frame}

\begin{frame}{Deployment}{Problem Analysis}
	\begin{itemize}
		\item Deployment was not done during this project.
		\item Mail 2.0
		\item Possible design mentioned Future Work.
	\end{itemize}
\end{frame}


%%%%%%%%%%%%%%%%%%%%%%%%%%%%%%%%%%%%%%%%%%%%%%%%%%%%%%%%%%%%%%%%%%%%%%%%%%%%%%%%
\section{Results}

\begin{frame}{Results}
	\begin{table}[H]
	  \centering
	  \begin{tabular}{l|l}
	    \textbf{Maximum Depth} & \textbf{Accuracy} \\ \hline
	    $4$                    & $92.84039$ \%     \\
	    $6$                    & $92.84846$ \%     \\
	    $8$                    & $92.75823$ \%
	  \end{tabular}
	  \caption{The results of the final datarun when training on the full training
	    set and trying to predict the entire test set.}
	  \label{tab:results00}
	\end{table}
\end{frame}

\begin{frame}{Results}
	\graphicc{1}{img/results_00}{The comditional inference tree produced by the
   		code when using a maximum depth of 4.}{fig:results00}
\end{frame}

\begin{frame}{Results}
	\begin{table}[H]
	  \centering
	  \begin{tabular}{l|l}
	    \textbf{Maximum Depth} & \textbf{Accuracy} \\ \hline
	    $4$                    & $91.79051$ \%     \\
	    $6$                    & $91.74852$ \%     \\
	    $8$                    & $91.74388$ \%
	  \end{tabular}
	  \caption{The results of the final data run when training on the full training
	    set excluding the \textit{logins14 variable} and trying to predict the
	    entire test set.}
	  \label{tab:results01}
	\end{table}
\end{frame}

\begin{frame}{Results}
	\graphicc{1}{img/results_01}{The conditional inference tree produced by the
    	code when using a maximum depth of 4 and exluding the \textit{logins14}
    	attribute.}{fig:results01}
\end{frame}

%%%%%%%%%%%%%%%%%%%%%%%%%%%%%%%%%%%%%%%%%%%%%%%%%%%%%%%%%%%%%%%%%%%%%%%%%%%%%%%%
\section{Conclusion and Future Work}

\begin{frame}{Conclusions and Future Work}
	\begin{itemize}
		%% Conclusions
		\item Getting the user to engange with the product is key for this classification target.
	    %% Future work
		\item New knowledge may be acquired by having more ``counter'' features.
		\item The produced code can be incorporated into a system that allows for automatic training and action.
	\end{itemize}
\end{frame}

%%%%%%%%%%%%%%%%%%%%%%%%%%%%%%%%%%%%%%%%%%%%%%%%%%%%%%%%%%%%%%%%%%%%%%%%%%%%%%%%
\section{Question}

\begin{frame}{Questions}
	\begin{itemize}
		\item Questions.
	\end{itemize}
\end{frame}



%%%%%%%%%%%%%%%%%%%%%%%%%%%%%%%%%%%%%%%%%%%%%%%%%%%%%%%%%%%%%%%%%%%%%%%%%%%%%%%%

%\begin{frame}[allowframebreaks=0.9]{References}
%\printbibliography
%\end{frame}

%%%%%%%%%%%%%%%%%%%%%%%%%%%%%%%%%%%%%%%%%%%%%%%%%%%%%%%%%%%%%%%%%%%%%%%%%%%%%%%%

\end{document}
