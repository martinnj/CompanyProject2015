\section{Conclusion}

After a product change Simplesite was facing a rapid influx of users for their
free services whom did not stay active on the site after 14 days of use. In
order to gather some knowledge, Simplesite whished to build a decision tree based
model that could be used to predict if a user would be retained after the 14
days.

This business problem shows strong reminisence to classic data science problems
such as analysing churn for a company. In order to leverage this, an overview of
the CRISP-DM process was achieved, this process was then followed and the steps
were described along with the results for each phase. This work includes both
experiments with different types of decision trees, different parameters for the
models and using a dataset that have been tweaked to contain the same amount of
retained and unretained customers.

With reference to the success criteria in Section \ref{sec:success}, work have
been done for all the entries on the list. A model have been created based on
the available data, and a number of R scripts have been written that can
recreate the model automatically and on request. Furthermore a number of tools
have been created for manually validating the quality of the model using
$k$-fold cross validation.

While Section \ref{sec:results} showed a number of different models, very little
was learned about the life-cycle of a retained customer as opposed to an
unretained customer. The most significant difference between the two is the
number of logins and edits the user performs. Based on this data it looks like
getting the customer to engage with the service regularly via logins and edits
is what greatly increases the probability of the customer being retained.

In Section \ref{sec:future} I also outlined a rough design for using the model
for an automated workflow, that could be used to automatically draw customer
data from the database once every other week. The older customer data can then
be used to determine if any of the new customers should be contacted to
``activate'' them. The script should place an email in an SQL based email queue
to be sent out to the relevant customers.

In the same section I also briefly described an idea for a new dataset that
might help the model become better at predicting the customers retention based
on their behaviour. The dataset would also help Simplesite gain a better
understanding of what separates retained and unretained customers by increasing
the accuracy of the model making it more representative for how customers use
the service.

Furthermore, a number of different parameters for the model was tested and the
results were documented for future reference. An attempt at manipulating the
dataset to obtain greater predictive accuracy for retained users was also
attempted, but ultimately was not useable due to lower mean accuracy.

The project is to serve as a prototype for how Simplesite can work with decision
trees and data collected from their customers, with a proper version to be
implemented by the companies own developers at a later time. The current
datasets are created from data logs Simplesite already had available, in the
future they will populate a dataset with event data extracted specifically for
the purpose of analysing the user behaviour. When this dataset is finished this
report and project can be used as a framework for how to work with the data.
