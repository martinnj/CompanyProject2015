\section{Problem Description}

Before 2014 123Hjemmeside offered paid subscription that allowed customers to
make and host a website using the provided CMS. Customers could get a time-
limited trial but would have to buy a subscription afterwards. If a cusotmer
stopped paying their subscription or did not wish to continue after their trial,
their website would be kept offline for a short while, but would be deleted
afterwards.

When 123hjemmeside changed its name to Simplesite in 2014 it also changed its
business model. They went from being purely paid to allowing everyone to create
as many websites as they wished for free, with no time limitation. The free
subscriptions give access to most functions of the CMS, excluding things like
personal domains, webshops and puts a limit on the number of images a website
can contain. In order to use these features a customer will have to sign up for
a paid subscription. This model of free and premium products is known as
freemium (\cite[p. 6]{bekkelund2011succeeding} and
\cite[p. 1]{pujol2010freemium}).

After the change of model Simplesite could see a huge increase in the number of
free websites being created, but not a corresponding increase in paid
subscriptions. They came up with the idea of recording how customers use the
site and use the data to build decision trees (\cite{breiman1984classification}
and \cite{quinlan1986induction}). The decision trees should not only be to be
able to classify customers based on behaviour, but also to let Simplesite learn
what events in the users life cycle that increase the chance of them becoming
paying customers.

Initially Simplesite is interested in exploring if there is any differences
between the users that still have logins 15 or more days after they are created
(Simplesite calls these customers \textit{retained}) and customers that simply
make a site and leave or forget about it. This mapping of important ``actions''
that some customers take is supposed to be used later in a mail procedure that
will automatically send out emails to new customers prompting them to perform
these actions that are known to cause customers to be retained.

% Simplesite changed their subscription service to a so called freemium model
% during 2014. This model means that everyone can have a website for free, and it
% will never be closed. This way of doing subscriptions means that more people
% sign up, and become potential customers, however a trend is that most customers
% that sign up either don't become paying customers, or simply make a site and
% stop using it after a few days.

% Simplesite wishes to map the life-cycle of customers in an attempt to find out
% how good users (users that are \textit{retained}, ie. have logins 3-4 weeks
% after they are created) use their site, as well as look at the life-cycles of
% the many free or abandoned customers. They hope to discover what, if any, the
% significant differences is in the different lifecycles. The goal is to attempt
% to guide new customers down the paths that are known to be ``good'' and
% hopefully detect if customers are stuck or have forgotten about their website.


%%%%%%%%%%%%%%%%%%%%%%%%%%%%%%%%%%%%%%%%%%%%%%%%%%%%%%%%%%%%%%%%%%%%%%%%%%%%%%%%
\subsection{Requirements}

Simplesite have already created some code in
R\footnote{\url{https://www.r-project.org/}} so any continued work should be
done in the R language. I have also been added to a Github repository containing
analytics code and a preferred structure and coding style that should be adhered
to.

Two datasets are also available containing different information gathered from
the live system. These two datasets should form the basis for all the
dataanalysis performed. An additional dataset is being constructed during the
project, but due to the time it takes to populate it with enough observations to
be meaningful it might not be finished before the project is over.


%%%%%%%%%%%%%%%%%%%%%%%%%%%%%%%%%%%%%%%%%%%%%%%%%%%%%%%%%%%%%%%%%%%%%%%%%%%%%%%%
\subsection{Success Criteria}

The project have 4 success criteria (that should all be fulfilled to some
degree):

\begin{enumerate*}
  \item A decision tree model for classifying customers created.
  \item New knowledge about customer lifecycles acquired from the model,
        in particular, do \textit{retained} customers have something in common.
  \item A prototype R script that can automatically build/create the model
        instance from new customer data should be created.
  \item A method for using the model should be designed or reasoned about.
\end{enumerate*}
