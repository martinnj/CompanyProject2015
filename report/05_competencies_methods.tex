%%%%%%%%%%%%%%%%%%%%%%%%%%%%%%%%%%%%%%%%%%%%%%%%%%%%%%%%%%%%%%%%%%%%%%%%%%%%%%%%
\subsection{Competencies and Methods}

In this section I will briefly reflect on how the skills and knowledge I have
acquired during my studies was used during the project as well as any skills I
have acquired during the project with Simplesite.

The major part of this project was rooted in the fields of Data Science and
Machine Learning. I had no experience or knowledge of Data Science but much of
the knowledge I have acquired through my studies was applicable since data
science uses a lot of techniques from other fields in computer science. The
CRISP-DM process from Figure \ref{fig:crisp} stems from data mining and is
strongly reminiscent of iterative software development, and data science
directly apply machine learning techniques such as model validation. I also
learned new things from studying data science, such as ``feature engineering'',
which is the dicipline of creating new features for a dataset using only the
existing features.

I also expanded my knowledge of machine learning with decision trees. It is a
model that was not covered during the ``Statistical Methods in Machine
Learning'' course at DIKU, but it is a very useful model to use if the model
should also be readable by humans.

Another skill I learned from DIKU that came in handy was the ability to quickly
pick up on a new programming language and learn to use it, since the project
required me to learn the R programming language and start using it immediately
due to the time constraints of the project. This particular ability is not
explicitly taught at DIKU, but most students pick it up as many courses
introduce a new programming language specifically for the course, with
``Advanced Programming'' using three different languages over a 2-3 month
period.

Two other skills that are not specifically tought at DIKU but most students pick
up are how to work with versioning and collaborative systems such as SVN or Git,
as well as the ability to quickly gather domain specific knowledge through
articles and textbooks. Both came into use during the project with Simplesite,
since the code is hosted on Github, and I had no existing knowledge on Data
Science and had to learn it from scratch.


% %%%%%%%%%%%%%%%%%%%%%%%%%%%%%%%%%%%%%%%%
% \subsubsection{Machine Learning}

% Machine learning forms the basis for creating the decision tree models and the
% model validation experiments I have performed. The skills is taught on the
% Masters programme during the course ``Stateistical Methods in Machine
% Learning''. I orignally did a lot of research into decision trees and how they
% are created before I was made aware that I would be using the R programming
% language, which have built-in packages for doing the model creation.

% My skills with machine learning increased slightly during the project since
% decision trees are not a method that was taught during the machine learning
% course.


% %%%%%%%%%%%%%%%%%%%%%%%%%%%%%%%%%%%%%%%%
% \subsubsection{Data Science}

% Data Science is a subject I did not yet know the name of when I started the
% project. Data Science is a field that covers several other diciplines such as
% data mining, machine learning and statistics, it uses tools from all these
% fields and use it to perform predictive anlysis based on large amounts of data
% in an attempt to find patterns and hidden knowledge.

% In this project I acquired some knowledge in this field during the project and
% used it to work with the data provided by Simplesite.


% %%%%%%%%%%%%%%%%%%%%%%%%%%%%%%%%%%%%%%%%
% \subsubsection{Advanced Programming}
% \todo{Find a better name for this skill.}

% One of the core things I learned from the ``Advanced Programming'' course at
% DIKU was to adapt to a new programming language in a short period of time, this
% particular skill was very needed for me to learn R and produce usable code
% within the approximately 2 month time limit of the project.


% %%%%%%%%%%%%%%%%%%%%%%%%%%%%%%%%%%%%%%%%
% \subsubsection{Knowledge Gathering}

% This is a skill that is not taught at DIKU but you develop as you take courses
% and get experience with finding litterature and seek the knowledge you will
% need. This is especially true for project courses since these usually will not
% have a specific curriculum.

% I improved the skill during this project when I had to find litterature for
% decision trees, R and Data Science.


% %%%%%%%%%%%%%%%%%%%%%%%%%%%%%%%%%%%%%%%%
% \subsubsection{Versioning / Collaborative Tools}

% Versioning and collaborative tools is another skill that is not taught
% explicitly during the programmes at DIKU, but it is something most people pick
% up during the project courses. During this project versioning have not played a
% big role, but since all the code is in a shared GitHub repository I have used it
% extensively to share code and ideas with Simplesite.
