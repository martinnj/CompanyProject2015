\documentclass[a4paper,10pt]{article}

%%%%%%%%%%%%%%%%%%%%%%%%%%%%%%%%%%%%%%%%%%%%%%%%%%%%%%%%%%%%%%%%%%%%%%%%%%%%%%%%
%%% Package selection
%%%%%%%%%%%%%%%%%%%%%%%%%%%%%%%%%%%%%%%%%%%%%%%%%%%%%%%%%%%%%%%%%%%%%%%%%%%%%%%%

% Language and encoding packages
\usepackage[utf8]{inputenc} % Accept utf8 encoded files.
\usepackage[T1]{fontenc}    % Text encoding.
\usepackage[english]{babel} % Mmultilingual for characters characters.

%%% Figure and table related packages
\usepackage{caption}    % Caption customization.
\usepackage{subcaption} % More caption tools.
\usepackage{float}      % Improved options for floats/figs.
\usepackage{minted}     % For syntax highlighting code.
\usepackage{tabularx}   % Allows us to use tables with \textwidth width.
%\usepackage{tikz}      % For creating diagrams.

%%% Misc Packages
\usepackage[lang=en,grid]{kufront} % KU Front page package.
\usepackage{todonotes}             % For enabling the \todo command.
\usepackage{mdwlist}               % For itemize* and enumerate* environments.
\usepackage{pdflscape}             % Allows landscape pages.
\usepackage{url}                   % Formatting for URLs.
\usepackage{a4wide}                % Decreases margin widths.

%%%%%%%%%%%%%%%%%%%%%%%%%%%%%%%%%%%%%%%%%%%%%%%%%%%%%%%%%%%%%%%%%%%%%%%%%%%%%%%%
%%% Type setting and own commands
%%%%%%%%%%%%%%%%%%%%%%%%%%%%%%%%%%%%%%%%%%%%%%%%%%%%%%%%%%%%%%%%%%%%%%%%%%%%%%%%

% What parts of the tikz package do we want to load?
%\usetikzlibrary{arrows,shapes,positioning,shadows,trees}

%Styleset for organization diagram.
% \tikzset{
%   basic/.style  = {draw, text width=2cm, drop shadow, font=\sffamily, rectangle},
%   board/.style   = {basic, rounded corners=2pt, thin, align=center,
%                    fill=white!30},
%   officer/.style = {basic, rounded corners=6pt, thin,align=center, fill=white!60,
%                    text width=8em},
%   department/.style = {basic, rounded corners=6pt, thin,align=center, fill=white!60,
%                    text width=8em},
%   employee/.style = {basic, thin, align=left, fill=pink!60, text width=6.5em}
% }

% Use sans-serif font for KU logo.
\renewcommand{\kufrontfont}{\sffamily}

% Redefine \theFancyVerbLine to enable line numbers in minted.
\renewcommand{\theFancyVerbLine}{\sffamily
\textcolor[rgb]{0.5,0.5,1.0}{\scriptsize
\oldstylenums{\arabic{FancyVerbLine}}}}

% Line used for the abstract.
% Usage: \HRule
\newcommand{\HRule}{\rule{\linewidth}{0.5mm}}

% Displays a centered image with caption and label.
% usage: \graphicc{width}{file}{caption}{label}
\newcommand{\graphicc}[4]{\begin{figure}[H] \centering
            \includegraphics[width={#1\textwidth}, keepaspectratio=true]{{#2}}
            \caption{{#3}} \label{#4} \end{figure}}

% Load some code from a file and display it in a pretty figure.
% usage: \codefig{language}{file}{firstline}{lastline}{caption}{label}
\newcommand{\codefig}[6]
{
\begin{figure}[H]
    \inputminted[firstnumber=#3,firstline=#3,lastline=#4,
                 linenos=true]{#1}{#2}
    \caption{#5 (#2)}
    \label{code:#6}
\end{figure}
}

%%%%%%%%%%%%%%%%%%%%%%%%%%%%%%%%%%%%%%%%%%%%%%%%%%%%%%%%%%%%%%%%%%%%%%%%%%%%%%%%
%%% Meta information for the KU front page.
%%%%%%%%%%%%%%%%%%%%%%%%%%%%%%%%%%%%%%%%%%%%%%%%%%%%%%%%%%%%%%%%%%%%%%%%%%%%%%%%

\title{User Behavior Analysis Using Decision Trees}
\author{\Large Martin Nicklas Jørgensen \quad {\ttfamily\large tzk173@alumni.ku.dk}}
\date{\today}
\project{\mdseries\LARGE Company Project \# 2015-1428}
\supervisor{Supervisor: Mikkel Rønne Jakobsen \quad {\ttfamily\large mikkelrj@di.ku.dk}}

%%%%%%%%%%%%%%%%%%%%%%%%%%%%%%%%%%%%%%%%%%%%%%%%%%%%%%%%%%%%%%%%%%%%%%%%%%%%%%%%
% Start the actual document.
%%%%%%%%%%%%%%%%%%%%%%%%%%%%%%%%%%%%%%%%%%%%%%%%%%%%%%%%%%%%%%%%%%%%%%%%%%%%%%%%

\begin{document}

% Create the front page.
\begin{titlepage}
\maketitle
\end{titlepage}

% Insert an abstract.
\hspace{1cm}\\[5cm]
{\huge Abstract}
\\\HRule\\

Simplesite is a danish web-hosting company based in Copenhagen. They recently
switched to a freemium based business model and stopped deleting inactive
websites entirely in a bid to have more users stay with them. This have lead to
a great influx of new customers using the free version of their services, many
of which open a free website with their service but do not stay active for very
long. Simplesite wishes to improve the situation by first getting users to stay
active for longer, and at a later date try to make the active users become
paying customers.

In order to gain a better understanding of what make some customers stay and
others not, a business problem is formulated and converted to a data science
problem; ``\textit{Is it possible to find any differences between customers who
are active on day 15 and later, and customers who leave before?}'' Through the
course of the report this problem is worked on using the Cross Industry Standard
Process for Data Mining (CRISP-DM); the business problem is converted into a
data science problem, I describe the available data and reason about
modifications to the datasets before creating and validating different models in
order to produce the best possible model. After the model have been created it
is evaluated and the process starts over.

The project went through several iterations of the CRISP-DM process, and each
iteration is covered and reasoned about. The results are evaluated and
suggestions for improvements and implementation ideas are outlined for future
work.

The models developed during the project show that for the currently available
data, two attributes are the most important for predicting if a customer is
retained or not; the number of logins and the number of edits during the first
14 days.

Finally the report contains my reflections on how my education have been used in
this company project and how the academic diciplines have I have been taught at
DIKU have been applied or expanded during the project.

\newpage

\tableofcontents

% actual report sections.
\section{Introduction}

During this project I work with the Danish website/hosting company Simplesite
ApS\footnote{\url{http://www.simplesite.com}} to help analyze user behaviour.
This analysis is done through the collection of behavioral data from a base of
existing customers and creating decision tree models based on this data.
Different types of decision trees and parameters was created and tested to find
the configuration that yields the best results and will allow Simplesite to gain
the best posbbile understanding about their customers behvaviours.

The report starts with information about Simplesite as a company as well as
their product. It then goes on to describe the business problem that I want to
solve. After the problem is described, the analysis section shows the Cross
Industry Standard Process for Data Mining\cite{shearer2000crisp} (CRISP-DM).
This process guides the entire process of understanding the business problem to
preparing data, creating models and evaluating and deploying the knowledge or
models acquired from the project.

In the end of the report I go over the results of the experiments and outline
som possible work to improve these results. I also reflect on how some of the
skills acquired at DIKU fit into the project as well as what new knowledge I
have acquired.

The source-code files from the project can be obtained by emailing me (see
frontpage for contact details).
                % Inroduction to the project.
\section{Simplesite ApS}


%%%%%%%%%%%%%%%%%%%%%%%%%%%%%%%%%%%%%%%%%%%%%%%%%%%%%%%%%%%%%%%%%%%%%%%%%%%%%%%%
\subsection{Product}

Simplesite produce and manage an inhouse website Content Management System (CMS)
and operate a hosting location where this CMS runs. Customers can register for a
free account allowing them to host a website that is edited through the CMS. The
free accounts are under restrictins on number fo pages, images and videos that
can be added to the site. Customers can then change to a paid subscription which
allows them to have more pages, images and videos on their website.

Simplesite also offer additional services to their customers that can be bought
for an extra fee if you are already a paying customer, this includes the ability
to have a domain attached to your website, a webshop as well as more additional
pages, images and videos.

The product is hosted partially on Simplesites own hardware in an offsite
location, and using a number of cloud services to provide faster response times
for certain data types.


%%%%%%%%%%%%%%%%%%%%%%%%%%%%%%%%%%%%%%%%%%%%%%%%%%%%%%%%%%%%%%%%%%%%%%%%%%%%%%%%
\subsection{Departments}

\begin{itemize*}
    \item \textbf{Administration} - Situated at the ground floor of the
          Copenhagen office, this department contains the HR functions as well
          as finance.

    \item \textbf{Sales} - Also on the ground floor of the Copenhagen office,
          sales consists of full time employeess and student helpers. The
          primary task is to sell the product, currently through localized ad
          management.

    \item \textbf{P \& C}\footnote{Product and Communication.} - The department
          sits on the upper floor of the Copenhagen office and is responsible
          for planning new features in cooperation with the developers as well
          as manage content on Simplesites own websites. P \& C also manages
          communication such as newsletters and localized ad texts.

    \item \textbf{In-House Development} - Sitting next to P \& C on the upper
          floor the developers are responsible for implementing new features as
          well as maintenence, analytics and bug fixing of the product.

    \item \textbf{Operations} - Daily operations are handled primarily by the
          company CTO, Thomas, as well as 2 part time students, operations sits
          on the upper floor in Copenhagen next to the developers.

    \item \textbf{Support} - Since the product is offered in several languages,
          a supporter is hired per language in a part time basis. All supporters
          work from home but get together once a month for a status meeting and
          to make sure new knowledge is shared and that relevant information can
          be given from the regular departments.

    \item \textbf{Remote Dev 1} - A small number of developers are hired in
          Bulgaria and have their own office in the Sofia, they are offered
          machines in the Copenhagen office they can VPN into and use for work.
          This allows the operations department to work from Copenhagen and
          still service the remote developers.

    \item \textbf{Remote Dev 2} - A number of developers are also hired from
          Serbia, like the remote developers in Bulgaria, they also VPN into the
          Copenhagen office and work on machines maintained by the regular
          operations employees.

    \item \textbf{Miscellaneous} - Simplesite also occasionally employs external
          specialists or consultants. Depending on the need, they will either
          work in the Copenhagen office or from some remote location using VPN.
\end{itemize*}


%%%%%%%%%%%%%%%%%%%%%%%%%%%%%%%%%%%%%%%%%%%%%%%%%%%%%%%%%%%%%%%%%%%%%%%%%%%%%%%%
\subsection{Company Structure}

\todo[inline]{Insert finished graphml figure.}
% \begin{landscape}
% \todo[inline]{Double check names and titles.}
% \begin{figure}[H]
% \centering
% \hspace*{-2cm}
% \begin{tikzpicture}[
%   level 1/.style={sibling distance=40mm},
%   edge from parent/.style={->,draw},
%   >=latex]

% % root of the the initial tree, level 1
% \node[board] {Board of Directors}
% % The first level, as children of the initial tree
%   child {node[officer] (c2) {\textbf{Thomas} - CTO}
%     child{node[department] (c21) {Operations}}
%     child{node[department] (c22) {Development}}
%   }
%   child {node[officer] (c1) {\textbf{Morten} - CEO}}
%   child {node[officer] (c3) {\textbf{Sebastian} - CSO}
%     child{node[department] (c31) {Sales}}
%   }
%   child {node[officer] (c5) {\textbf{Lars} - Product and Communication Manager}
%     child{node[department] (c51) {P \& C}}
%   }
%   child[missing]
%   child {node[officer,xshift=-2cm] (c4) {\textbf{Pia} - CFO}
%     child{node[department] (c41) {Administration}}
%     child{node[department] (c42) {Support}}
%   };

% \end{tikzpicture}

% \caption{Organizational diagram for Simplesite ApS}
% \label{fig:orgdiagram}
% \end{figure}
% \end{landscape}     % Description of the company structure and workflow.
\section{Problem Description}

Simplesite changed their subscription service to a so called freemium model
during 2015. This model means that everyone can have a website for free, and it
will never be closed. This way of doing subscriptions means that more people
sign up, and become potential customers, however a trend is that most customers
that sign up either don't become paying customers, or simple make a site and
stop using it after a few days.

Simplesite wishes to map the life-cycle of customers in an attempt to find out
how paying (good) users use their site, as well as look at the life-cycles of
the many free or abandoned customers. They hope to discover what, if any, the
significant differences is in the different lifecycles. The goal is to attempt
to guide new customers down the paths that are known to be ``good'' and hopefully
detect if customers are stuck or have forgotten about their website.

\subsection{Requirements}
\todo[inline]{This is garbage, fix it.}
Simplesite have already started working on creating datasets and attempting to
map variables. In order work with their existing code I was given access to a
code repository which I can commit to.

The existing code is written in R\footnote{\url{https://www.r-project.org/}} and
as such, the code I will be producing will also be written in R. Furthermore, in
order to easy human understanding of the data models worked out, decision trees
(and related machinelearning models) was chosen as the model. Decision trees,
once constructed offer an intuitive way to look at what parameters lead to
different outcomes. They also closely resemble a number of customer life-cycles
combined into a graph.
         % Description of the problem.
\section{Problem Analysis}


%%%%%%%%%%%%%%%%%%%%%%%%%%%%%%%%%%%%%%%%%%%%%%%%%%%%%%%%%%%%%%%%%%%%%%%%%%%%%%%%
\subsection{Available Data}

The basis for the analysis is 2 datasets created by Simplesite:
\texttt{EngagementData} \& \texttt{CustomerJourney}. Table
\ref{tab:engdatalayout} and Table \ref{tab:cjdatalayout} names the features of
each dataset and what they represent. Both datasets contain users created
between September 1st 2015 and September 15 2015. \todo{This window should maybe
be expanded a little for more data.}

\begin{table}[H]
	\centering
	\begin{tabularx}{\textwidth}{l|X}
		\textbf{Attribute Name} & \textbf{Attribute Data}                                                                                                     \\ \hline
		\textit{islogins1}                       & Bool, true if: one or more logins for the user.                                                            \\
		\textit{islogins2}                       & Bool, true if: two or more logins for the user.                                                            \\
		\textit{islogins3}                       & Bool, true if: three or more logins for the user.                                                          \\
		\textit{islogins4}                       & Bool, true if: four or more logins for the user.                                                           \\
		\textit{isedit30m}                       & Bool, true if: User edited site within 30 minutes of creation.                                             \\
		\textit{isedit24h}                       & Bool, true if: User edited site within 24 hours of creation (excluding the first 30 minutes).              \\
		\textit{isaddpage30m}                    & Bool, true if: User added a new page within 30 minutes of creation.                                        \\
		\textit{isaddpage24h}                    & Bool, true if: User added a new page within 24 hours of creation (excluding the first 30 minutes).         \\
		\textit{isimgupload30m}                  & Bool, true if: User uploaded their own image within 30 minutes of creation.                                \\
		\textit{isimgupload24h}                  & Bool, true if: User uploaded their own image within 24 hours of creation (excluding the first 30 minutes). \\
		\textit{iseditdesign30m}                 & Bool, true if: User edited site within 30 minutes of creation.                                             \\
		\textit{iseditdesign24h}                 & Bool, true if: User edited site within 24 hours of creation (excluding the first 30 minutes).              \\
		\textit{customerid}                      & Integer value with the customers unique ID.                                                                \\
		\textit{marketname}                      & String with the market the user came from (US, TR, DK etc.)                                                \\
		\textit{siteverkey}                      & String with what version of the site the user is created in (US, TR, DK etc.)                              \\
		\textit{ispayer}                         & Bool, true if: The customer have a paid subscription.                                                      \\
		\textit{culturekey}                      & String with language information for the site (en-US, fr-FR etc.)                                          \\
		\textit{iso14}                           & Bool used by marketing.
	\end{tabularx}
	\caption{Features found in the \texttt{EngagementData} dataset.}
	\label{tab:engdatalayout}
\end{table}


\begin{table}[H]
	\centering
	\begin{tabularx}{\textwidth}{l|X}
		\textbf{Attribute Name} & \textbf{Attribute Data}                                                                      \\ \hline
		\textit{customerid}     & Integer value with the customers unique ID.                                                  \\
		\textit{logins14}       & Integer, number of times the customer logged in the first 14 days (week 1-2 after creation). \\
		\textit{logisnw2w4}     & Integer, number of times the customer logged in in week 3-4 after creation.                  \\
		\textit{edits14}        & Integer, number of times the customer edited a page within the first 14 days.                \\
		\textit{iscjtrial}      & Bool, true if: Always true, everyone starts as a trial.                                      \\
		\textit{iscjonboarded}  & Bool, true if: edits14 $\geq 1$.                                                             \\
		\textit{iscjactivated}  & Bool, true if: edits14 $\geq 3$.                                                             \\
		\textit{iscjengaged}    & Bool, true if: edits14 $\geq 6$ and logins14 $\geq 2$.                                       \\
		\textit{iscjinvested}   & Bool, true if: edits14 $\geq 15$ and logins14 $\geq 6$.                                      \\
		\textit{iscjretained}   & Bool, true if: logisnw2w4 $\geq 1$.                                                          \\
		\textit{isimgupload1d}  & Bool, true if: Customer uploaded an image within the first 24 hours of being created.        \\
		\textit{iseditdesign1d} & Bool, true if: Customer edited the design within the first 24 hours of being created.        \\
		\textit{isaddpage1d}    & Bool, true if: Customer added a new page within the first 24 hours of being created.         \\
		\textit{isedit1d}       & Bool, true if: Customer edited a page within the first 24 hours of being created.           
	\end{tabularx}
	\caption{Features found in the \texttt{CustomerJourney} dataset.}
	\label{tab:cjdatalayout}
\end{table}


%%%%%%%%%%%%%%%%%%%%%%%%%%%%%%%%%%%%%%%%%%%%%%%%%%%%%%%%%%%%%%%%%%%%%%%%%%%%%%%%
\subsection{Pruning Datasets}

The initial goal is to find customers who are retained (\textit{iscjretained} =
\texttt{True}) and see if there is some pattern, that Simplesite can try to
guide other customers down in order to increase the number of retained
customers. With this in mind there is some attributes of the datasets that will
not be helpful, either because they cannot be controlled/changed, or because
they do not make sense. The following is a list of attributes removed from the
\texttt{EngeagementData} dataset during work, along with the reason for the
removal.

\begin{itemize*}
  \item \textit{islogins1} : Removed since it is always true for all customers.

	\item \textit{islogins2} : Removed because the definition of a retained
	      customer requires one or more logins, so this must always be true.

	\item \textit{islogins3} : Same as \textit{islogins2}.

	\item \textit{islogins4} : Same as \textit{islogins2}.

	\item \textit{marketname} : Removed since we are unable to get a customer from
	      a different market, we are insterested in variable we can change for
	      each customer.

	\item \textit{siteverkey} : Same as \textit{marketname}.

	\item \textit{ispayer} : Removed because it is an alternative target variable,
	      it does not say anything about how the user behaves, other than they are
	      indeed a good customer.

	\item \textit{culturekey} : \textit{marketname}.

	\item \textit{iso14} : Value used by marketing. \todo{is o14 usefull? Clear up with Morten.}
\end{itemize*}

\noindent The following is a list of attributes removed from the \texttt{CustomerJourney}
dataset during work, along with the reason for the removal.

\begin{itemize*}
	\item \textit{logins14} : Removed because initial tests showed high bias. For
	      a ctree\footnote{Conditional Inference Tree.} of depth 4, the three top
	      levels was occupied with choices regarding logins14.

	\item \textit{logisnw2w4} : Removed since this attribute is in the definition
	      of our target variable \textit{iscjretained}.

	\item \textit{iscjtrial} : Removed since it is always true.

	\item \textit{iscjonboarded} : Removed since it serves as an alternative target
	      variable, and is set by us, it does not say anything about the user
	      behaviour that is not already present.

	\item \textit{iscjengaged} : Same as \textit{iscjonboarded}.
	\item \textit{iscjinvested} : Same as \textit{iscjonboarded}.
\end{itemize*}

In both datasets the \textit{customerid} attribute is kept in each dataset, even
though it cannot be used as a feature for analysis since each customer have a
unique ID and thus will not yield any patterns, since it can be used to join the
two datasets together.
     % Analysis and solution of the problem area.
%%%%%%%%%%%%%%%%%%%%%%%%%%%%%%%%%%%%%%%%%%%%%%%%%%%%%%%%%%%%%%%%%%%%%%%%%%%%%%%%
\section{Results}
\label{sec:results}

This section will contain the results of the project, both in term of the final
models created for classifying whether or not a customer will be retained, but
also recommendations for future work that could improve the solution.


%%%%%%%%%%%%%%%%%%%%%%%%%%%%%%%%%%%%%%%%
\subsection{Modelling Results}

In Section \ref{sec:eval} we saw that when removing the bias towards unretained
observations from the dataset, the overall accuracy of the model suffers
severly, with that  in mind, I chose to train the future models on the full
dataset, since this dataset also contains what the user have already ``done'',
even if a customer is mistakenly identifyed as a user that will not be retained,
Simplesite can check via this data that they do not accidentally send out emails
with tips and tricks that the customer already know.

By using the full training dataset and formula with the highest precision from
Table \ref{tab:formulacompare} I have run the tests by training a model on the
entire dataset (witout cross validation) and tested the resulting model on the
full test set. In order to make sure that the model did not overfit, I tested
with three different maximum depths to see if there would be any significant
difference due to overfitting, the code for running this experiment can be found
in Appendix \ref{app:results} Figure \ref{app:code:results}.

\begin{table}[H]
  \centering
  \begin{tabular}{l|l}
    \textbf{Maximum Depth} & \textbf{Accuracy} \\ \hline
    $4$                    & $92.84039$ \%     \\
    $6$                    & $92.84846$ \%     \\
    $8$                    & $92.75823$ \%
  \end{tabular}
  \caption{The results of the final datarun when training on the full training
    set and trying to predict the entire test set.}
  \label{tab:results00}
\end{table}

Table \ref{tab:results00} shows that the difference between allowing the tree to
grow $6$ levels deep, compared to $4$ in terms of accuracy is less than one
hunredth of a percent. The results also show that allowing the tree to grow to
$8$ levels deep causes a slight loss of precision that is most likely due to an
overfitting for the training data.

Due to the size of the trees I was only able to include the tree with a maximum
depth of $4$ in the report, since the larger ones does not fit on A4 paper or on
my screen\footnote{Displaying them on a $2560 \times 1440$p screen results in
nodes and leaves overlapping.} but they can be reproduced by running the code
from Figure \ref{app:code:results}. Figure \ref{fig:results00} shows the
resulting model when drawn on the screen. Each node in the drawing corresponds
to an attribute, or a choice, and each edge out of the nodes corresponds to a
value or statement about the attribute. The leafs contain 2 pieces of
information; the $n$ variable which is the number of observations in the
training set that belongs to this leaf, as well as the propability of the
observations in that leaf to be false (unretained). If we use the leftmost leaf
as an example, there is $275873$ observations that belong in this leaf, and the
propability that any of the observations that land there is unretained is $1$,
with a $0$ propability that they are retained.

\begin{landscape}
  \graphicc{1.5}{img/results_00}{The comditional inference tree produced by the
    code when using a maximum depth of 4.}{fig:results00}
\end{landscape}

Figure \ref{fig:results00} also shows that the \textit{logins14} variable is
taking up a lot of the nodes. This could indicate (based on our data) that one
of the most important factors in retaining a user, is to make them form the
habbit of logging in. Users that log in more than $12$ times during the first 14
days have a propability of at least $0.6$ to be retained.

In order to learn if any other variables can impact the customer retention I
performed a test with the \textit{logins14} attribute removed from both the
training and test datasets. (This moves us back into the data preparation phase
of the CRISP-DM process). The accuracy data from the test can be seen in Table
\ref{tab:results01}.

\begin{table}[H]
  \centering
  \begin{tabular}{l|l}
    \textbf{Maximum Depth} & \textbf{Accuracy} \\ \hline
    $4$                    & $91.79051$ \%     \\
    $6$                    & $91.74852$ \%     \\
    $8$                    & $91.74388$ \%
  \end{tabular}
  \caption{The results of the final data run when training on the full training
    set excluding the \textit{logins14 variable} and trying to predict the
    entire test set.}
  \label{tab:results01}
\end{table}

The result of excluding \textit{logins14} is a drop in accuracy of around $1$
\%, but when looking at the tree produced we might gain more knowledge. The tree
produced by limiting the depth to $4$ can be seen in Figure \ref{fig:results01}.

\begin{landscape}
  \graphicc{1.5}{img/results_01}{The conditional inference tree produced by the
    code when using a maximum depth of 4 and exluding the \textit{logins14}
    attribute.}{fig:results01}
\end{landscape}

Looking at Figure \ref{fig:results01} we can see a lot more of the attributes
have been selected, but also that the propability of retention seem to be rather
low in most of the leafs. The 4 leafs with the best retention span from $0.3$ to
$0.7$ and all come from users that have more than $16$ edits within the first 14
days of their time with Simplesite.

Based on the two resulting trees in Figure \ref{fig:results00} and Figure
\ref{fig:results01}, it is hard to pinpoint a series of specific actions that
makes retained/retainable users easy to see. The two most significant traits (as
we could see by the numbers in Table \ref{tab:formulacompare}) seems to be the
number of logins as well as edits during the first 14 days of the users time
with Simplesite.

%%%%%%%%%%%%%%%%%%%%%%%%%%%%%%%%%%%%%%%%
\subsection{Continued Work}
\label{sec:future}


\subsubsection{Dataset}

In order to further improve the models a new dataset could be compiled, based on
this project it looks like ``counter'' attributes give a lot more information
than boolean attributes; a dataset that also counts the number of image uploads
or edits within a more fine grained time window might shed more light on the
behvaiour differences of retained and unretained customers.


\subsubsection{Automation}

This project includes code that allows an R program to load data from a clear-
text file and use it to automatically create a model from the dataset.
Simplesite already have code that allows the programs to pull the data directly
from the database, so as long as the script is run regularly it can pull fresh
data and create an up-to-date model.

A similar approach could be used for uploading the predictions to a database
after the model is created and the new user data have been acquired. Since the
model is part of a new program to send out more personalized emails to
customers, it can be used to find customers that have very low login or edit
counts, or are somehow deviating from the patterns we see in retained customers.
When the model classifies a customer that can be prompted by an email, it can use
the same SQL driver it used to get user data to add the customer to the email
queue.

The following procedure illustrates this idea:
\begin{enumerate*}
  \item Get a training set of customers that are at least a month old.
  \item Get a set of customers to predict on, these could be created 14 days ago
    and to the date of this execution.
  \item Create a conditional inference tree model based on the training set.
  \item Use the model to predict which of the new customers are ``on the right track''.
  \item Discard these customers, we don't wish to interfere with them.
  \item For the remaining customers in the list: based on knowledge from the
    database, add them to the email queue and send them an appropriate email,
    for example: ``Did you know you can create a free image gallery on your
    Simplesite?'' or similar.
\end{enumerate*}

The implementation and regular use of this procedure would hopefully lead to
increased retention of customers with free accounts, and with some adoption and
a change of the target variable also be able to predict on whether or not a
customer will become apaying customer. This model could be applied at a later
stage so new users are fed to the current model that predicts retention, and
customers that are retained can be fed into a model that predict on becoming a
paying customer, further expanding the idea of guiding potential customers along
a behavior that is known to have a high propability of becoming a paying
customer.
              % Selected solution and work done. Includes results of work.
%%%%%%%%%%%%%%%%%%%%%%%%%%%%%%%%%%%%%%%%%%%%%%%%%%%%%%%%%%%%%%%%%%%%%%%%%%%%%%%%
\subsection{Competencies and Methods}

In this section I will briefly reflect on how the skills and knowledge I have
acquired during my studies was used during the project as well as any skills I
have acquired during the project with Simplesite.

The major part of this project was rooted in the fields of Data Science and
Machine Learning. I had no experience or knowledge of Data Science but much of
the knowledge I have acquired through my studies was applicable since data
science uses a lot of techniques from other fields in computer science. The
CRISP-DM process from Figure \ref{fig:crisp} stems from data mining and is
strongly reminiscent of iterative software development, and data science
directly apply machine learning techniques such as model validation. I also
learned new things from studying data science, such as ``feature engineering'',
which is the dicipline of creating new features for a dataset using only the
existing features.

I also expanded my knowledge of machine learning with decision trees. It is a
model that was not covered during the ``Statistical Methods in Machine
Learning'' course at DIKU, but it is a very useful model to use if the model
should also be readable by humans.

Another skill I learned from DIKU that came in handy was the ability to quickly
pick up on a new programming language and learn to use it, since the project
required me to learn the R programming language and start using it immediately
due to the time constraints of the project. This particular ability is not
explicitly taught at DIKU, but most students pick it up as many courses
introduce a new programming language specifically for the course, with
``Advanced Programming'' using three different languages over a 2-3 month
period.

Two other skills that are not specifically tought at DIKU but most students pick
up are how to work with versioning and collaborative systems such as SVN or Git,
as well as the ability to quickly gather domain specific knowledge through
articles and textbooks. Both came into use during the project with Simplesite,
since the code is hosted on Github, and I had no existing knowledge on Data
Science and had to learn it from scratch.


% %%%%%%%%%%%%%%%%%%%%%%%%%%%%%%%%%%%%%%%%
% \subsubsection{Machine Learning}

% Machine learning forms the basis for creating the decision tree models and the
% model validation experiments I have performed. The skills is taught on the
% Masters programme during the course ``Stateistical Methods in Machine
% Learning''. I orignally did a lot of research into decision trees and how they
% are created before I was made aware that I would be using the R programming
% language, which have built-in packages for doing the model creation.

% My skills with machine learning increased slightly during the project since
% decision trees are not a method that was taught during the machine learning
% course.


% %%%%%%%%%%%%%%%%%%%%%%%%%%%%%%%%%%%%%%%%
% \subsubsection{Data Science}

% Data Science is a subject I did not yet know the name of when I started the
% project. Data Science is a field that covers several other diciplines such as
% data mining, machine learning and statistics, it uses tools from all these
% fields and use it to perform predictive anlysis based on large amounts of data
% in an attempt to find patterns and hidden knowledge.

% In this project I acquired some knowledge in this field during the project and
% used it to work with the data provided by Simplesite.


% %%%%%%%%%%%%%%%%%%%%%%%%%%%%%%%%%%%%%%%%
% \subsubsection{Advanced Programming}
% \todo{Find a better name for this skill.}

% One of the core things I learned from the ``Advanced Programming'' course at
% DIKU was to adapt to a new programming language in a short period of time, this
% particular skill was very needed for me to learn R and produce usable code
% within the approximately 2 month time limit of the project.


% %%%%%%%%%%%%%%%%%%%%%%%%%%%%%%%%%%%%%%%%
% \subsubsection{Knowledge Gathering}

% This is a skill that is not taught at DIKU but you develop as you take courses
% and get experience with finding litterature and seek the knowledge you will
% need. This is especially true for project courses since these usually will not
% have a specific curriculum.

% I improved the skill during this project when I had to find litterature for
% decision trees, R and Data Science.


% %%%%%%%%%%%%%%%%%%%%%%%%%%%%%%%%%%%%%%%%
% \subsubsection{Versioning / Collaborative Tools}

% Versioning and collaborative tools is another skill that is not taught
% explicitly during the programmes at DIKU, but it is something most people pick
% up during the project courses. During this project versioning have not played a
% big role, but since all the code is in a shared GitHub repository I have used it
% extensively to share code and ideas with Simplesite.
 % Competencies and methods.
\section{Conclusion}

After a product change Simplesite was facing a rapid influx of users for their
free services whom did not stay active on the site after 14 days of use. In
order to gather some knowledge, Simplesite whished to build a decision tree based
model that could be used to predict if a user would be retained after the 14
days.

This business problem shows strong reminisence to classic data science problems
such as analysing churn for a company. In order to leverage this, an overview of
the CRISP-DM process was achieved, this process was then followed and the steps
were described along with the results for each phase. This work includes both
experiments with different types of decision trees, different parameters for the
models and using a dataset that have been tweaked to contain the same amount of
retained and unretained customers.

With reference to the success criteria in Section \ref{sec:success}, work have
been done for all the entries on the list. A model have been created based on
the available data, and a number of R scripts have been written that can
recreate the model automatically and on request. Furthermore a number of tools
have been created for manually validating the quality of the model using
$k$-fold cross validation.

While Section \ref{sec:results} showed a number of different models, very little
was learned about the life-cycle of a retained customer as opposed to an
unretained customer. The most significant difference between the two is the
number of logins and edits the user performs. Based on this data it looks like
getting the customer to engage with the service regularly via logins and edits
is what greatly increases the probability of the customer being retained.

In Section \ref{sec:future} I also outlined a rough design for using the model
for an automated workflow, that could be used to automatically draw customer
data from the database once every other week. The older customer data can then
be used to determine if any of the new customers should be contacted to
``activate'' them. The script should place an email in an SQL based email queue
to be sent out to the relevant customers.

In the same section I also briefly described an idea for a new dataset that
might help the model become better at predicting the customers retention based
on their behaviour. The dataset would also help Simplesite gain a better
understanding of what separates retained and unretained customers by increasing
the accuracy of the model making it more representative for how customers use
the service.

Furthermore, a number of different parameters for the model was tested and the
results were documented for future reference. An attempt at manipulating the
dataset to obtain greater predictive accuracy for retained users was also
attempted, but ultimately was not useable due to lower mean accuracy.

The project is to serve as a prototype for how Simplesite can work with decision
trees and data collected from their customers, with a proper version to be
implemented by the companies own developers at a later time. The current
datasets are created from data logs Simplesite already had available, in the
future they will populate a dataset with event data extracted specifically for
the purpose of analysing the user behaviour. When this dataset is finished this
report and project can be used as a framework for how to work with the data.
           % Final conclusion.

\end{document}
